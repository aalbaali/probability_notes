\section{Problem statement}
How can convergence of a random process $\rv{x}$ or a random sequence (RS)\index{Random sequqnce} $\rv{x}_{n}$ be defined?

The case is not as straightforward as in the deterministic case. Therefore, there are different types of convergence in for random processes (sequences). Specifically, 
\begin{enumerate}
    \item convergence \emph{everywhere} (\emph{e}) or \emph{surely} (\emph{s}),
    \item convergence \emph{almost everywhere} (\emph{a.e.}) or \emph{almost surely} (\emph{a.s.}), or \emph{with probability 1} (\emph{w.p.~1})
    \item convergence in \emph{probability} (\emph{p}),
    \item convergence in \emph{mean square sense} (\emph{m.s.}), and
    \item convergence in \emph{distribution}.
\end{enumerate}
The focus in this chapter will be on random sequences but the concepts generalize to random processes.

\begin{myBlueBox}
    \textbf{Notation:}
    Let $\rv{x}_{n}$ be a random sequence. Then a realization of $\rv{x}_{n}$ is given by
    \begin{align}
        x_{n}(s), \qquad s\in S,
    \end{align}
    or simply $x_{n}$.
\end{myBlueBox}
\section{Definitions}
