\section{The maximum a-posteriori estimator}
In this section, the MAP estimator for normally distributed variables will be derived. For each problem, a process model is provided in the form 
\begin{align}
\mbf{x}_{k+1} &= \mbf{f}(\mbf{x}_{k}, \mbf{u}_{k}, \mbfuline{w}_{k}),
\end{align}
where $\mbfuline{w}_{k}\sim\mc{N}(\mbf{0},\mbs{\Sigma}_{\mbf{w}})$ is the Gaussian process noise.
Further, a measurement model is given in the form
\begin{align}
\mbf{y}_{k} &= \mbf{g}(\mbf{x}_{k}, \mbfuline{v}_{k}),
\end{align}
where $\mbfuline{v}_{k}\sim\mc{N}(\mbf{0},\mbs{\Sigma}_{\mbf{v}})$ is the Gaussian measurement noise. 

The goal is to estimate the states $\mbf{x}_{1:K}$ given the measurements and their distributions $\mbf{u}_{0:K-1}$, $\mbf{y}_{1:K}$, and the prior on the initial state $\mbf{x}_{0}\sim\mc{N}(\mbfhat{x}_{0},\mbfhat{P}_{0})$. To derive the maximum a-posteriori (MAP) estimator, the likelihood of $\mbf{x}_{1:K}$ must first be derived. 

The posterior of $\mbf{x}_{1:K}$ is given by
\begin{align}
\ell\left(\mbf{x}_{1:K}; \mbf{u}_{0:K-1}, \mbf{y}_{1:K}, \mbfhat{x}_{0}\right)
&= \pdf{\mbf{x}_{1:K} \middle\vert~ \mbf{u}_{0:K-1}, \mbf{y}_{1:K}, \mbfhat{x}_{0}}\\
&= 
\f{
    \pdf{\mbf{y}_{1:K} \middle\vert~ \mbf{u}_{0:K-1}, \mbfhat{x}_{0}, \mbf{x}_{1:K}}\pdf{\mbf{x}_{1:K}\middle\vert~ \mbf{u}_{0:K-1}, \mbfhat{x}_{0}}
}
{
    \pdf{\mbf{y}_{1:K}\middle\vert~\mbf{u}_{0:K-1}, \mbfhat{x}_{0}}
}\\
&=
    \label{eq: LS derivation Bayes. Marginalizing x, finding eta}
    \f{
        \pdf{\mbf{y}_{1:K} \middle\vert~ \mbf{u}_{0:K-1}, \mbfhat{x}_{0}, \mbf{x}_{1:K}}\pdf{\mbf{x}_{1:K}\middle\vert~ \mbf{u}_{0:K-1}, \mbfhat{x}_{0}}
    }
    {
        \int_{-\infty}^{\infty} \pdf{\mbf{y}_{1:K}\middle\vert~\mbf{u}_{0:K-1}, \mbfhat{x}_{0}, \mbf{x}_{1:K}}\pdf{\mbf{x}_{1:K}}\dee\mbf{x}_{1:K}
    }\\
&= 
    \eta \pdf{\mbf{y}_{1:K} \middle\vert~ \mbf{u}_{0:K-1}, \mbfhat{x}_{0}, \mbf{x}_{1:K}}\pdf{\mbf{x}_{1:K}\middle\vert~ \mbf{u}_{0:K-1}, \mbfhat{x}_{0}}\\
&= 
    \label{eq: LS derivation Bayes. y independent of u}
    \eta \pdf{\mbf{y}_{1:K} \middle\vert~ \mbf{x}_{1:K}}\pdf{\mbf{x}_{1:K}\middle\vert~ \mbf{u}_{0:K-1}, \mbfhat{x}_{0}}\\
&= 
    \label{eq: LS derivation Bayes. y independent of ys}
    \eta \prod_{k=1}^{M}\pdf{\mbf{y}_{k} \middle\vert~ \mbf{x}_{k}}\pdf{\mbf{x}_{1:K}\middle\vert~ \mbf{u}_{0:K-1}, \mbfhat{x}_{0}}\\
&= 
    \label{eq: LS derivation Bayes. Markov prop}
    \eta \prod_{k=1}^{M}\pdf{\mbf{y}_{k} \middle\vert~ \mbf{x}_{k}}
    \prod_{k=1}^{K}\pdf{\mbf{x}_{k}\middle\vert~ \mbf{x}_{k-1}, \mbf{u}_{k-1}}\pdf{\mbf{x}_{0}}\\
&= 
    \label{eq: LS derivation Bayes. Reorganize}
    \eta
    \pdf{\mbf{x}_{0}}
    \prod_{k=1}^{K}\pdf{\mbf{x}_{k}\middle\vert~ \mbf{x}_{k-1}, \mbf{u}_{k-1}}
    \prod_{k=1}^{M}\pdf{\mbf{y}_{k} \middle\vert~ \mbf{x}_{k}}\\
&= 
    \label{eq: LS derivation Bayes. Expand with exponential}
    \eta
    \gaussian{\mbf{x}_{0}}{\mbfhat{x}_{0}}{\mbfhat{P}_{0}}
    \prod_{k=1}^{K}\gaussian{\mbf{x}_{k}}{\mbf{f}\left(\mbf{x}_{k-1}, \mbf{u}_{k-1}\right)}{\mbs{\Sigma}_{\mbf{w}}}
    \prod_{k=1}^{M}\gaussian{\mbf{y}_{k}}{\mbf{g}\left(\mbf{x}_{k}\right)}{\mbs{\Sigma}_{\mbf{v}}},
\end{align}
where $p(\cdot)$ is the probability distribution function (PDF) of a random variable and $\eta$ is the normalizing coefficient given in the denominator of \eqref{eq: LS derivation Bayes. Marginalizing x, finding eta}. Equation~\eqref{eq: LS derivation Bayes. y independent of u} is obtained by taking into account that $\mbf{y}_{k}$ is independent of $\mbf{u}_{k}$ for all $k$, \eqref{eq: LS derivation Bayes. y independent of ys} is obtained because $\mbf{y}_{j}$ is independent of $\mbf{y}_{i}$ for $i\neq j$, \eqref{eq: LS derivation Bayes. Markov prop} exploits the Markov property of the process model, and \eqref{eq: LS derivation Bayes. Expand with exponential} is the Gaussian PDF written out.

Taking the negative log of \eqref{eq: LS derivation Bayes. Expand with exponential} gives
\begin{align}
    \label{eq: LS derivation. -Log likelihood}
    L\left(\mbf{x}_{1:K};  \mbf{u}_{0:K-1}, \mbf{y}_{1:K}, \mbfhat{x}_{0}\right) &= 
    -\log\ell\left(\mbf{x}_{1:K};  \mbf{u}_{0:K-1}, \mbf{y}_{1:K}, \mbfhat{x}_{0}\right)\\
    &=
    \eta \left(
    \f{1}{2} \norm{\mbf{x}_{0}-\mbfhat{x}_{0}}_{\mbfhat{P}_{0}\inv}^{2} +
    \f{1}{2}\sum_{k=1}^{K}\norm{\mbf{x}_{k} - \mbf{f}\left(\mbf{x}_{k-1}, \mbf{u}_{k-1}\right)}_{\mbs{\Sigma}_{\mbf{w}}\inv}^{2} \right.\\\nonumber&\qquad\left. + 
    \f{1}{2}\sum_{k=1}^{M}\norm{\mbf{y}_{k} - \mbf{g}\left(\mbf{x}_{k}\right)}_{\mbs{\Sigma}_{\mbf{v}}\inv}^{2}
    \right),
\end{align}
The \emph{maximum a-posteriori} (MAP) estimate, as the name suggests, finds the $\mbf{x}$ that maximizes the posterior \eqref{eq: LS derivation Bayes. Expand with exponential} which is \emph{equivalent} to minimizing the negative log \eqref{eq: LS derivation. -Log likelihood}. 
% However, the MLE is not desirable as it requires the computation of $\eta$ which is expensive. However, since
    Since $\eta$ does not depend on $\mbf{x}$, then it does not affect the optimization problem. Maximizing the negative-log posterior \eqref{eq: LS derivation. -Log likelihood} with respect to $\mbf{x}$ without computing $\eta$ gives the MAP estimator. Specifically,
\begin{align}
    \mbfhat{x}_{0:K, \mathrm{MAP}} &= \argmin_{\mbf{x}_{1:K}} \f{1}{2} \norm{\mbf{x}_{0}-\mbfhat{x}_{0}}_{\mbfhat{P}_{0}\inv}^{2} +
    \f{1}{2}\sum_{k=1}^{K}\norm{\mbf{x}_{k} - \mbf{f}\left(\mbf{x}_{k-1}, \mbf{u}_{k-1}\right)}_{\mbs{\Sigma}_{\mbf{w}}\inv}^{2} +
    \f{1}{2}\sum_{k=1}^{M}\norm{\mbf{y}_{k} - \mbf{g}\left(\mbf{x}_{k}\right)}_{\mbs{\Sigma}_{\mbf{v}}\inv}^{2}
\end{align}
which is a least squares problem.



%%%%%%%%%%%%%%%%%%%%%%%%%%%%%%%%%%%%%%%%%%%%%%%%%%%%%%%%%%%%%%%%%%%%%%%%%%%%%%%%%%%%%%%%%%%%%%
% Conditional probability
%%%%%%%%%%%%%%%%%%%%%%%%%%%%%%%%%%%%%%%%%%%%%%%%%%%%%%%%%%%%%%%%%%%%%%%%%%%%%%%%%%%%%%%%%%%%%%
\section{Passing measurements through a function}
\begin{mytheorem}
   [Passing measurements through a function]    
   Let $\mbfrv{y}\in\rnums^{m}$ be a (vector) random variable that depends on $\mbfrv{x}\in\rnums^{n}$ through
   \begin{align}
       \mbfrv{y} &= \mbf{g}\left( \mbfrv{x} \right).
   \end{align}
   Furthermore, let $\mbfrv{z}\in\rnums^{p}$ be a transformed random variable given by
   \begin{align}
       \mbfrv{z} &= \mbf{f}\left( \mbfrv{y} \right).
   \end{align}
   Then 
   \begin{align}
       \pdf{\mbfrv{x}\middle|~\mbf{y}, \mbf{z}} 
       &=
       \pdf{\mbfrv{x}\middle|~\mbf{y}}.
    \end{align}
    Furthermore, if $\mbf{g}$ is invertible and $\mbf{f}$ is invertible, then 
    \begin{align}
        \label{eq:conditional prob xyz eq xz}
        \pdf{\mbfrv{x}\middle|~\mbf{y}, \mbf{z}} 
        &=
        \pdf{\mbfrv{x}\middle|~\mbf{z}}.
    \end{align}
    Note that if $\mbf{g}$ is invertible but $\mbf{f}$ is non-invertible, then \eqref{eq:conditional prob xyz eq xz} does not hold in general!
\end{mytheorem}
\begin{proof}
    \begin{enumerate}
        \item The first part.
        \begin{align}
            \pdf{\mbf{x}\middle|~\mbf{y}, \mbf{z}} 
            &=
            \f{
                \pdf{\mbf{y}, \mbf{z}\middle|~\mbf{x}}\pdf{\mbf{x}}
            }{
                \pdf{\mbf{y}, \mbf{z}}
            }\\
            &= 
            \f{
                \pdf{\mbf{z}\middle|~\mbf{y},\mbf{x}}\pdf{\mbf{y}\middle|~\mbf{x}}\pdf{\mbf{x}}
            }{
                \pdf{\mbf{z}\middle|~\mbf{y}}\pdf{\mbf{y}}
            }\\
            &= 
            \f{
                \cancel{\pdf{\mbf{z}\middle|~\mbf{y}}}\pdf{\mbf{y}\middle|~\mbf{x}}\pdf{\mbf{x}}
            }{
                \cancel{\pdf{\mbf{z}\middle|~\mbf{y}}}\pdf{\mbf{y}}
            }\\
            &= 
                \f{
                    \pdf{\mbf{y}\middle|~\mbf{x}}\pdf{\mbf{x}}
                }{
                    \pdf{\mbf{y}}
                }\\
            &= \pdf{\mbf{x}\middle|~\mbf{y}}.
        \end{align}

        \item The second part. Let's expand $\pdf{\mbf{x}\middle|~\mbf{y},\mbf{z}}$.
        \begin{align}
            \pdf{\mbf{x}\middle|~\mbf{y}, \mbf{z}}
            &=
            \f{
                \pdf{\mbf{y},\mbf{z}\middle|~\mbf{x}}\pdf{\mbf{x}}
            }{
                \pdf{\mbf{y}, \mbf{z}}
            }\\
            &=
            \f{
                \pdf{\mbf{y}\middle|~\mbf{z},\mbf{x}}\pdf{\mbf{z}\middle|~\mbf{x}}\pdf{\mbf{x}}
            }{
                \pdf{\mbf{y}\middle|~\mbf{z}}\pdf{\mbf{z}}
            }\\
            &=
            \f{
                \pdf{\mbf{y}\middle|~\mbf{z},\mbf{x}}
            }{
                \pdf{\mbf{y}\middle|\mbf{z}}
            }
            \pdf{\mbf{x}\middle|~\mbf{z}}.
        \end{align}
        Thus, \eqref{eq:conditional prob xyz eq xz} satisfied if
        \begin{align}
            \pdf{\mbf{y}\middle|~\mbf{z},\mbf{x}} &= \pdf{\mbf{y}\middle|\mbf{z}}.
        \end{align}
        That is, if $\mbf{y}$ is independent of $\mbf{x}$ \emph{given} $\mbf{z}$. More rigorously, if there exists $\mbf{y}_{1}\neq\mbf{y}_{2}$ such that
        \begin{align}
            \begin{aligned}
                \mbf{g}(\mbf{x}_{1}) &= \mbf{y}_{1}\\
                \mbf{g}(\mbf{x}_{2}) &= \mbf{y}_{2}\\
            \end{aligned}
            &\implies \mbf{x}_{1}\neq\mbf{x}_{2},
        \end{align}
        and
        \begin{align}
            \label{eq:conditional probl proof z=fy1=fy2}
            \mbf{z} = \mbf{f}\left( \mbf{y}_{1} \right) 
            = \mbf{f}\left( \mbf{y}_{2} \right).
        \end{align}
        Then knowing $\mbf{z}$ without knowing $\mbf{x}$ will result in a non-singleton sample set $S$ for the random variable $\mbfrv{y}$. That is, a set with more than one element which implies that $\mbfrv{y}$ cannot be determined uniquely. The sample space is the set of $\mbf{y}$ that satisfy \eqref{eq:conditional probl proof z=fy1=fy2}.

        On the other hand, if in addition to knowing $\mbf{z}$, $\mbf{x}$ is also known, then the sample space of $\mbf{y}$ is a singleton. That is, $\mbfrv{y}$ can be determined uniquely.

        The condition \eqref{eq:conditional probl proof z=fy1=fy2} occurs only if the function $\mbf{f}$ is non-invertible.
    \end{enumerate}
\end{proof}
