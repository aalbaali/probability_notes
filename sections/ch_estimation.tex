\section{Motivation}
Let $\rv{x}$ be a random variable whose CDF, $\cdf{x}$, \emph{is unknown}. The goal is estimate $\cdf{x}$ from $n$ observations of $\rv{x}$, $x_{1}, \ldots, x_{n}$.

\begin{mydefinition}[Empirical CDF]
    The empirical CDF of $\rv{x}$ from the observations $x_{1}, \ldots, x_{n}$ is
    \begin{align}
        \hat{\cdf{}}(x) &= \f{1}{n} \times \left( \# \text{ of observations } x_{i} \leq x \right).
    \end{align}
\end{mydefinition}
The empirical CDF $\hat{\cdf{}}(x)$ is the CDF of a discrete random variable with PMF
\begin{align}
    \hat{\pmf{}}_{n}(x) &= 
    \begin{cases}
        \f{1}{n}, & x = x_{i}, \quad i = 1,\ldots, n,\\
        0 & \text{Otherwise}.
    \end{cases}
\end{align}
The PDF is then given by
\begin{align}
    \label{eq:empirical pdf constant weight}
    \hat{\pdf{}}_{n}(x) &= \sum_{i=1}^{n}\f{1}{n}\delta(x-x_{i}).
\end{align}

The empirical PDF $\hat{\pdf{}}(x)$ in \eqref{eq:empirical pdf constant weight} assigns constant weights to every observation. The empirical PDF can be generalized by giving different weights to different observations. Specifically,
\begin{align}
    \hat{\pdf{}}_{n}(x) &= \sum_{i=1}^{n}\f{1}{n}k(x-x_{i}),
\end{align}
where $k(\cdot)$ is the kernel function such that it is
\begin{enumerate}
    \item symmetric around 0, and
    \item $\int_{-\infty}^{\infty}k(x)\dee x = 1$.
\end{enumerate}
A common choice is the Gaussian kernel. That is, a PDF of $\mc{N}(0,1)$.


%%%%%%%%%%%%%%%%%%%%%%%%%%%%%%%%%%%%%%%%%%%%%%%%%%%%%%%%%%%%%%%%%%%%%%%%%%%%%%%%%%%%
% Parameter estimation
%%%%%%%%%%%%%%%%%%%%%%%%%%%%%%%%%%%%%%%%%%%%%%%%%%%%%%%%%%%%%%%%%%%%%%%%%%%%%%%%%%%%
\section{Problem statement}
Let $\mbfrv{x}$ be a random variable. Furthermore, let $\pdf{\mbf{x}; \mbs{\theta}}$ be the PDF of $\mbfrv{x}$ of \emph{known form} (e.g., Gaussian distribution) but depending on \emph{unknown} parameters $\mbs{\theta}$ (e.g., mean and covariance of a Gaussian distribution).
\begin{example}
    A Gaussian random variable has a PDF parameterized w.r.t. to the mean ${\mu}$ and the variance $\sigma^{2}$. Thus, the parameters could be $\mbs{\theta} = \bbm \mu, \sigma^{2} \ebm$.
\end{example}

The goal is to estimate $\mbs{\theta}$ from $n$ observations of $\mbfrv{x}$: $\mbf{x}_{1}, \ldots, \mbf{x}_{n}$. 

Define the data vector 
\begin{align}
    \mbf{X} &= \bbm \mbf{x}_{1}\\\vdots\\\mbf{x}_{n} \ebm.
\end{align}
The data vector $\mbf{X}$ is a realization of the random vector 
\begin{align}
    \mbfrv{X} &= \bbm \mbfrv{x}_{1} \\ \vdots \\ \mbfrv{x}_{n} \ebm,
\end{align}
where $\mbfrv{x}_{i}$, $i=1,\ldots, n$ are i.i.d.

\subsection*{Approaches}
There are two approaches to parameter estimation.
\begin{enumerate}
    \item \emph{Point estimation}: identify a function $g(\cdot)$ of the sample such that $g(\mbfrv{x}_{1},\ldots, \mbfrv{x}_{n})$ is a good (in some sense) approximation of $\mbs{\theta}$.
    \item \emph{Interval estimation}: identify two functions $g_{1}(\cdot)$ and $g_{2}(\cdot)$ of the sample, such that
    \begin{align}
        \prob{g_{1}(\mbfrv{x}_{1},\ldots,\mbfrv{x}_{n})<\mbs{\theta}<g_{2}(\mbfrv{x}_{1},\ldots,\mbfrv{x}_{n})} = \gamma,
    \end{align}
    where $\gamma$ is the confidence interval.
\end{enumerate}

\section{Definitions and terminology}
\begin{mydefinition}[Estimator]
    The estimator of $\mbs{\theta}$ is given in the form
    \begin{align}
        \hat{\mbsrv{\theta}} &= g(\mbfrv{x}_{1}, \ldots, \mbfrv{x}_{n}).
    \end{align}

    The estimator models the behaviour of the estimation method over all possible samples.
\end{mydefinition}
\begin{mydefinition}[Estimate of $\mbs{\theta}$]
    The \emph{deterministic} quantity $g(\mbf{x}_{1},\ldots,\mbf{x}_{n})$. That is, the the output of $\hat{\mbsrv{\theta}}$ for the realizations $\mbf{x}_{1}, \ldots, \mbf{x}_{n}$.
\end{mydefinition}
\begin{mydefinition}[Statistic]
    Any function of $\mbfrv{x}_{1}, \ldots, \mbfrv{x}_{n}$.
\end{mydefinition}
\begin{mydefinition}[Bias]
    The estimator $g(\mbfrv{x}_{1}, \ldots, \mbfrv{x}_{n})$ is called
    \begin{itemize}
        \item \emph{unbiased} if 
        \begin{align}
            \expect{g(\mbfrv{x}_{1},\ldots, \mbfrv{x}_{n})} &= \mbs{\theta}.
        \end{align}
        \item Otherwise, it's called \emph{biased} with bias
        \begin{align}
            \expect{g(\mbfrv{x}_{1},\ldots, \mbfrv{x}_{n})} - \mbs{\theta}.
        \end{align}
    \end{itemize}
\end{mydefinition}

\begin{mydefinition}[Asymptic unbias]
    A biased estimator for which
    \begin{align}
        \expect{g(\mbfrv{x}_{1},\ldots,\mbfrv{x}_{n})} \overset{N\to\infty}{\to} \mbs{\theta}
    \end{align}
    is called \emph{asymptotically unbiased}.
\end{mydefinition}

\begin{mydefinition}[Mean-squared best estimator]
   For a fixed $n$, the estimator that minimizes the mean-squared (MS) error 
   \begin{align}
       \mbf{e}_{n} &= \expect{\abs{g(\mbfrv{x}_{1},\ldots\mbfrv{x}_{n})^{\trans}g(\mbfrv{x}_{1},\ldots\mbfrv{x}_{n})}}
   \end{align}
   is called the \emph{best} estimator (in the mean-squared sense).
\end{mydefinition}

\begin{mydefinition}[Consistency]
    The estimator $g(\mbfrv{x}_{1},\ldots\mbfrv{x}_{n})$ is called \emph{consistent} if 
    \begin{align}
        \lim_{n\to\infty} \prob{\abs{g(\mbfrv{x}_{1},\ldots\mbfrv{x}_{n})}>\mbs{\epsilon}} &= 0, \forall \mbs{\epsilon}>\mbf{0},
    \end{align}
    or
    \begin{align}
        \lim_{n\to\infty} \prob{\abs{g(\mbfrv{x}_{1},\ldots\mbfrv{x}_{n})}<\mbs{\epsilon}} &= 1, \forall \mbs{\epsilon}>\mbf{0}.
    \end{align}
\end{mydefinition}

\begin{mytheorem}
    [Sufficient condition for consistency]    
    Let $g(\mbfrv{x}_{1},\ldots\mbfrv{x}_{n})$ be an estimator of $\mbs{\theta}$. Then, if
    \begin{align}
        \lim_{n\to\infty} \expect{\abs{g(\mbfrv{x}_{1},\ldots\mbfrv{x}_{n})^{\trans}g(\mbfrv{x}_{1},\ldots\mbfrv{x}_{n})}} &= \mbf{0},
    \end{align}
    then $g(\mbfrv{x}_{1},\ldots\mbfrv{x}_{n})$ is a \emph{consistent estimator} of $\mbs{\theta}$.

    The converse is not true in general. That is, this is a sufficient but not necessary condition.
\end{mytheorem}


\begin{mytheorem}
   [Sample mean estimator]    
     Let $\mbfrv{x}_{1},\ldots, \mbfrv{x}_{n}$ be a random sample of size $n$ and let $\mbfrv{x}_{1},\ldots, \mbfrv{x}_{n}$ be \emph{uncorrelated}. The \emph{sample mean estimator}
     \begin{align}
         \hat{\mbsrv{\mu}}_{n} &= \f{1}{n}\sum_{i=1}^{n}\mbfrv{x}_{i}
     \end{align}
     is
     \begin{itemize}
         \item an \emph{unbiased estimator} of $\mbs{\mu}$;
         \item a \emph{consistent estimator} of $\mbs{\mu}$.
     \end{itemize}
\end{mytheorem}

\begin{mytheorem}
   [Variance estimaton]    
   Consider the sample average estimators of the mean and variance of Gaussian random variable $\rv{x}$ from a random sample $\rv{x}_{1},\ldots, \rv{x}_{n}$ of size $n$:
   \begin{align}
       \hat{\rv{\mu}}_{n} &= \f{1}{n}\sum_{i=1}^{n}\rv{x}_{i},\\
       \hat{\rv{\sigma}}_{n}^{2} &= \f{1}{n} \sum_{i=1}^{n}\left( \rv{x}_{i}-\hat{\rv{\mu}}_{n} \right)^{2}.
   \end{align}
   The random variables $\hat{\rv{\mu}}_{n}$ and $\hat{\rv{sigma}}_{n}^{2}$ are independent.
\end{mytheorem}

\section{Method of moments}
Recall that moments are defined in Def.~\ref{def:moments single rv} and Def.~\ref{def:moments multiple rv}. 
The $m$-th moment estimator is given by
\begin{align}
    \hat{m}_{m}(\mbf{X}) &= 
    \f{1}{n}\sum_{i}^{n}x_{i}^{m}.
\end{align}
Let $\mbs{\theta}\in\rnums^{k}$ be of size $k$ and let 
\begin{align}
    \mbf{X} &= \bbm x_{1} \\\vdots\\x_{n} \ebm 
\end{align}
be the samples of the random variable $\rv{x}$. Then, 
the idea of method of moments is to estimate the first $k$ moments, thus getting $k$ equations, and then solving the $k$ equations for $\mbs{\theta}$.

\begin{example}
    Estimate the mean $\mu$ and variance $\sigma^{2}$ of a random variable $\rv{x}$ from a random sample $\mbfrv{X} = \bbm \rv{x}_{1}&\ldots&\rv{x}_{n} \ebm^{\trans}$.

    The moments are given by
    \begin{alignat}{2}
        m_{1}(\mu, \sigma^{2}) &= \expect{\rv{x}} &&= \mu, \\
        m_{2}(\mu, \sigma^{2}) &= \expect{\rv{x}^{2}} &&= \mu^{2} + \sigma^{2}.
    \end{alignat}
    The sample average estimates of the moments are given by
    \begin{align}
        \hat{m}_{1}(\mbf{X}) &= \f{1}{n}\sum_{i=1}^{n}x_{i},\\
        \hat{x}_{2}(\mbf{X}) &= \f{1}{n}\sum_{i=1}^{n}x_{i}^{2}.
    \end{align}
    Thus, the parameter estimates can be computed by solving the system of equations
    \begin{align}
        \mu &= \hat{m}_{1}(\mbf{X})\\
        \mu^{2} + \sigma^{2} &= \hat{m}_{2}(\mbf{X}).
    \end{align}
    The solution is given by
    \begin{align}
        \hat{\mu} &= \hat{m}_{1}(\mbf{X}) \\
            &= \f{1}{n}\sum_{i=1}^{n}x_{i},\\
        \hat{\sigma}^{2} &= \hat{m}_{2}(\mbf{X}) - \hat{m}_{1}^{2}(\mbf{X})\\
        &= \f{1}{n}\sum_{i=1}^{n}x_{i}^{2}  - \left(\f{1}{n}\sum_{i=1}^{n}x_{i}\right)^{2}.
    \end{align}
\end{example}


%%%%%%%%%%%%%%%%%%%%%%%%%%%%%%%%%%%%%%%%%%%%%%%%%%%%%%%%%%%%%%%%%%%%%%%%%%%%%%%%%%%%
% Maximum likelihood
%%%%%%%%%%%%%%%%%%%%%%%%%%%%%%%%%%%%%%%%%%%%%%%%%%%%%%%%%%%%%%%%%%%%%%%%%%%%%%%%%%%%
\section{Maximum likelihood (ML) estimator}
\begin{mydefinition}[Likelihood]
    Given the observations $\mbfrv{x}_{1} = \mbf{x}_{1}', \ldots, \mbfrv{x}_{n}=\mbf{x}_{n}'$, the \emph{likelihood} of $\mbs{\theta}$ is the joint PDF of $\mbfrv{x}_{1},\ldots,\mbfrv{x}_{n}$ evaluated at the observed points
    \begin{align}
        L(\mbs{\theta}; \mbf{x}_{1}', \ldots, \mbf{x}_{N}') &=
        \pdf{\mbf{x}_{1}',\ldots, \mbf{x}_{n}';\mbs{\theta}}\\
         &= \pdf{}_{\mbf{x}}(\mbf{x}_{1}';\mbs{\theta})\cdots\pdf{}_{\mbf{x}}(\mbf{x}_{n}';\mbs{\theta}).
    \end{align}
\end{mydefinition}
\begin{myremark}
    \begin{itemize}
        \item The likelihood $\pdf{\mbf{x}_{1}',\ldots, \mbf{x}_{n}';\mbs{\theta}}$ is treated as a function of $\mbs{\theta}$.
        \item It is not a PDF. That is,
        \begin{align}
            \int_{-\infty}^{\infty}\pdf{\mbf{x}_{1}',\ldots, \mbf{x}_{n}';\mbs{\theta}}\dee\mbs{\theta}\neq 1.
        \end{align}
    \end{itemize}
\end{myremark}

\begin{mydefinition}[Maximum-likelihood estimator]
    The maximum-likelihood (ML) estimate of $\mbs{\theta}$, $\hat{\mbs{\theta}}_{\mathrm{ML}}$, is the value of $\mbs{\theta}$ that maximizes $\pdf{\mbf{x}_{1}',\ldots, \mbf{x}_{n}';\mbs{\theta}}$. That is,
    \begin{align}
        \hat{\mbs{\theta}}_{\mathrm{ML}} 
        &= \argmax_{\mbs{\theta}\in\rnums^{k}} L(\mbs{\theta}; \mbf{x}_{1}', \ldots, \mbf{x}_{N}')\\
        &= \argmax_{\mbs{\theta}\in\rnums^{k}} \pdf{\mbf{x}_{1}',\ldots, \mbf{x}_{n}';\mbs{\theta}}.
    \end{align}
\end{mydefinition}

\begin{myremark}
    Some remarks regarding the ML estimator.
    \begin{itemize}
        \item Log-likelihood function 
        \begin{align}
            \ell(\mbf{x}_{1}',\ldots,\mbf{x}_{n}';\mbs{\theta}) 
            &= \log L(\mbs{\theta}; \mbf{x}_{1}', \ldots, \mbf{x}_{N}')\\
            &= \log \pdf{\mbf{x}_{1}',\ldots, \mbf{x}_{n}';\mbs{\theta}}.
        \end{align}
        \item The maximizer of the log-likelihood function is the same maximizer of the likelihood function. That is,
        \begin{align}
            \argmax_{\mbs{\theta}\in\rnums^{k}} L(\mbs{\theta}; \mbf{x}_{1}', \ldots, \mbf{x}_{N}')
            &= \argmax_{\mbs{\theta}\in\rnums^{k}} \ell(\mbs{\theta}; \mbf{x}_{1}', \ldots, \mbf{x}_{N}').
        \end{align}
        \item The ML estimate is 
        \begin{align}
            \hat{\mbs{\theta}}_{\mathrm{ML}} 
            &= \argmax_{\mbs{\theta}\in\rnums^{k}} \pdf{\mbf{x}_{1}',\ldots, \mbf{x}_{n}';\mbs{\theta}}\\
            &= \argmax_{\mbs{\theta}\in\rnums^{k}} \log \pdf{\mbf{x}_{1}',\ldots, \mbf{x}_{n}';\mbs{\theta}}\\
            &= \argmax_{\mbs{\theta}\in\rnums^{k}} L(\mbf{x}_{1}',\ldots,\mbf{x}_{n}';\mbs{\theta}).
        \end{align}
    \end{itemize}
\end{myremark}

\begin{myremark}
    Some remarks on ML estimation.
    \begin{enumerate}
        \item Let $\hat{\mbsrv{\theta}}_{\mathrm{ML}}$ be the ML estimator of $\mbs{\theta}$ and let $g(\mbs{\theta})$ be a function of $\mbs{\theta}$. Then, the ML estimator of $g(\mbs{\theta})$ is $g(\hat{\mbsrv{\theta}}_{\mathrm{ML}})$.
        
        \item Limitations of the ML estimation:
        \begin{enumerate}
            \item The ML estimate may not exist.
            \item The ML estimate may not be unique. 
        \end{enumerate}        
        
        \item Maximum likelihood estimate can be finding by minimizing the negative likelihood or negative log-likelihood.
    \end{enumerate}
\end{myremark}
    
\begin{example}
    Let $\rv{x}\sim U[0,\theta]$ be a uniformly distributed random variable in the form
    \begin{align}
        \pdf{x; \theta} &= 
        \begin{cases}
            \f{1}{\theta}, &x\in[0,\theta],\\
            0,&\text{otherwise},
        \end{cases}
    \end{align}
    where $\theta$ is some unknown parameter. Find the ML estimator of $\theta$ from a random sample of size $N$.

    \begin{enumerate}
        \item First, get the joint PDF of all the random samples
        \begin{align}
            \pdf{x_{1}, \ldots, x_{N}; \theta} &= \pdf{x_{1};\theta}\cdots\pdf{x_{N}; \theta}\\
            &= 
            \begin{cases}
                \f{1}{\theta^{N}}, &x_{i}\in[0, \theta], i=1,\ldots,N,\\
                0, &\text{otherwise},
            \end{cases}
        \end{align}
        where the fact that the random samples $\rv{x}_{i}$ for $i=1,\ldots,N$ are i.i.d. 

        \item The likelihood of $\theta$ is then given as a function of the samples of $\rv{x}_{i}=x_{i}'$. Specifically, the likelihood is
        \begin{align}
            L(\theta; x_{1}', \ldots, x_{N}') &=
            \pdf{\rv{x}_{1} = x_{1}', \ldots, \rv{x}_{N} = x_{N}'; \theta}\\
            &=
            \begin{cases}
                \f{1}{\theta^{N}}, &x_{i}'\in[0,\theta], i=1,\ldots, N, \\
                0, &\text{otherwise}
            \end{cases}\\
            &=
            \begin{cases}
                \label{eq:example ML uniform distribution}
                \f{1}{\theta^{N}}, &\max_{i}(x_{i})\leq\theta, \\
                0, &\text{otherwise}.
            \end{cases}
        \end{align}
        Note that it was not necessary to specify the lower bound $\min_{i}(x_{i})\geq 0$ since the samples are drawn from a uniform distribution $U[0, \theta]$ which is bounded from below by $0$. Thus, $\prob{\rv{x}<0} = 0$. Therefore, it is impossible to have a sample $x_{i}'<0$.

        \item Maximize the likelihood function. Maximizing \eqref{eq:example ML uniform distribution} is a \emph{constrained} optimization problem that entails maximizing $\f{1}{\theta^{N}}$. 
        
        Maximizing $\f{1}{\theta^{N}}$ is the same as minimizing $\theta$. However, since $\theta$ is bounded from below by the constraint in \eqref{eq:example ML uniform distribution}, then the minimum value of $\theta$ is $\max_{i}(x_{i})$. Thus,
        \begin{align}
            \hat{\theta}_{\mathrm{ML}} &= \max_{i}(x_{i}).
        \end{align}
    \end{enumerate}
    \triqed
\end{example}
\section{Bayesian parameter estimation}
% \subsection*{Introduction and terminology}
In the previous sections, the approaches used were the \emph{classical} or \emph{frequentist} approaches. The classical approach
\begin{enumerate}
    \item treats the unknown parameter as deterministic parameter, and
    \item assumes no prior information on the unknown parameter was available.
\end{enumerate}

For Bayesian approach on the other hand
\begin{enumerate}
    \item assumes that unknown parameter is random with known pdf $\pdf{\mbs{\theta}}$, called the \emph{prior};
    \item the JPDF of random sample given $\mbsrv{\theta} = \theta$ (likelihood) is given by
    \begin{align}
        \pdf{\mbf{x}_{1},\ldots,\mbf{x}_{n}\middle|~ \mbs{\theta}};
    \end{align}
    \item The estimation problem becomes a prediction problem; predict $\mbsrv{\theta}$ from a correlated (with $\mbsrv{\theta}$) sample $\mbfrv{X}$.
\end{enumerate}

The mean-squared (or Bayes) estimate $\hat{\mbs{\theta}}$ of $\mbsrv{\theta}$ given $\mbfrv{X}=\mbf{X}$ is
\begin{align}
    \label{eq:bayesian estimator E(theta|X)}
    \hat{\mbs{\theta}} &= \expect{\mbsrv{\theta}\middle|~\mbf{X}}\\
    &= \int_{-\infty}^{\infty}\mbs{\theta}\pdf{\mbs{\theta}\middle|~\mbf{X}}\dee\mbs{\theta},
\end{align}
where
\begin{align}
    \pdf{\mbs{\theta}\middle|~\mbf{X}} &= 
        \f{
            \pdf{\mbf{X}\middle|~\mbs{\theta}}\pdf{\mbs{\theta}}
        }{\pdf{\mbf{X}}}\\
    &=
        \f{
            \pdf{\mbf{X}\middle|~\mbs{\theta}}\pdf{\mbs{\theta}}
        }{
            \int_{-\infty}^{\infty}\pdf{\mbf{X}\middle|~\mbs{\theta}}\pdf{\mbs{\theta}}\dee\mbs{\theta}
        }.
\end{align}

\begin{myBlueBox}
    \textbf{Terminology}
    \begin{itemize}
        \item The mean-squared estimator is also referred to as the Bayes estimator.
        \item $\pdf{\mbs{\theta}}$: \emph{prior} pdf of $\mbsrv{\theta}$ (before observing the sample).
        \item $\pdf{\mbs{\theta}\middle|~\mbf{X}}$: \emph{posterior} pdf of $\mbs{\theta}$ (after observing the sample).
        \item In \cite{barfoot_state_2017}, the term $\pdf{\mbf{X}\middle|~\mbs{\theta}}$ is sometimes referred to as the likelihood of $\mbf{X}$ given $\mbs{\theta}$. This is not to be confused with the likelihood from the ML estimator. In estimation, the term $\pdf{\mbf{X}\middle|~\mbs{\theta}}$ can also be referred to as an \emph{observation model}. 
    \end{itemize}
\end{myBlueBox}

\begin{myremark}
    Some remarks about Bayesian estimation.    
    \begin{itemize}
        \item $\expect{\mbsrv{\theta}\middle|~\mbf{X}}$ from \eqref{eq:bayesian estimator E(theta|X)} is difficult to solve for two reasons. First, the denominator of the posterior pdf involves evaluating an integral over the entire sample space of $\mbs{\theta}$. Second, evaluating the expectation $\expect{\cdot}$ requires \emph{another} integral to be evaluated over the entire sample space.
        
        \item Maximum a-posteriori (MAP) estimate maximizes the \emph{posterior} pdf $\pdf{\mbs{\theta}\middle|~\mbf{X}}$ with respect to $\mbs{\theta}$. Since the denominator of the prior is constant, then it can be omitted. That is,
        \begin{align}
            \hat{\mbs{\theta}}_{\mathrm{MAP}} 
            &= \argmax_{\mbs{\theta}} \pdf{\mbs{\theta}\middle|~\mbf{X}}\\
            &= \argmax_{\mbs{\theta}} \f{\pdf{\mbf{X}\middle|~\mbs{\theta}}\pdf{\mbs{\theta}}}{\pdf{\mbf{X}}}\\
            &= \argmax_{\mbs{\theta}} \pdf{\mbf{X}\middle|~\mbs{\theta}}\pdf{\mbs{\theta}}.
        \end{align}
    \end{itemize}
\end{myremark}

\section{Maximum a-posteriori (MAP) estimator}
This estimator is an \emph{approximation} to the Bayes estimator discussed in the previous section. It is easier (less computationally intensive) to compute.
\begin{mydefinition}[Maximum a-posteriori estimator]
    Maximum a-posteriori (MAP) estimate maximizes the \emph{posterior} pdf $\pdf{\mbs{\theta}\middle|~\mbf{X}}$ with respect to $\mbs{\theta}$. Since the denominator of the prior is constant, then it can be omitted. That is,
        \begin{align}
            \hat{\mbs{\theta}}_{\mathrm{MAP}} 
            &= \argmax_{\mbs{\theta}} \pdf{\mbs{\theta}\middle|~\mbf{X}}\\
            &= \argmax_{\mbs{\theta}} \f{\pdf{\mbf{X}\middle|~\mbs{\theta}}\pdf{\mbs{\theta}}}{\pdf{\mbf{X}}}\\
            &= \argmax_{\mbs{\theta}} \pdf{\mbf{X}\middle|~\mbs{\theta}}\pdf{\mbs{\theta}}.
        \end{align}
\end{mydefinition}

\subsection{MAP estimator of a Markov normally distributed random variable}

In this section, the MAP estimator for normally distributed variables that follow a Markov chain will be derived. For each problem, a process model is provided in the form 
\begin{align}
    \label{eq:MAP state estimation process model}
    \mbfrv{x}_{k+1} &= \mbfrv{f}(\mbf{x}_{k}, \mbfrv{u}_{k}, \mbfuline{w}_{k}),
\end{align}
where $\mbfuline{w}_{k}\sim\mc{N}(\mbf{0},\mbf{Q}_{k})$ is the process noise and the covariance matrix $\mbf{Q}_{K}$ is known. A prior is distributed according to $\mbfhat{x}_{0}\sim\mc{N}\left( \mbfhat{x}_{1}, \mbfcheck{P}_{0} \right)$.

Furthermore, a measurement model is given in the form
\begin{align}
    \label{eq:MAP state estimation measurement model}
    \mbfrv{y}_{k} &= \mbf{g}(\mbfrv{x}_{k}, \mbfuline{v}_{k}),
\end{align}
where $\mbfuline{v}_{k}\sim\mc{N}(\mbf{0},\mbf{R}_{k})$ is the Gaussian measurement noise, where the covariance matrix $\mbf{R}_{k}$ is known. 

The goal is to estimate states' $\mbfrv{x}_{1:K}$ parameters using the realizations of the prior $\mbf{x}_{0}$, the realizations of the interoceptive measurements $\mbf{u}_{k}$ and realizations of the exteroceptive measurements $\mbf{y}_{k}$. If the RVs are Gaussian, then they can be parametrized by a mean and a covariance (i.e., $\mbfrv{x}_{k}\sim\mc{N}(\mbs{\mu}_{k}, \mbs{\Sigma}_{k})$). 

\begin{myremark}
    The interoceptive measurement $\mbfrv{u}_{k}$ is a random variable that can be assumed to be distributed according to
    \begin{align}
        \mbfrv{u}_{k}\sim\mc{N}\left( \mbf{u}_{k}, \mbf{Q}_{k}^{\mbf{u}} \right).
    \end{align}
    The measurement is assumed to be random due to measurement noise. However, the measurement noise can be embedded into the process noise $\mbf{Q}$. Therefore, it is possible to assume that the interoceptive measurements are not random, but the interoceptive measurement process noise is indeed random and will be embedded with the process noise $\mbfrv{w}_{k}$.
\end{myremark}
\begin{myremark}
    The states will be assumed to be normally distributed. That is, $\mbfrv{x}_{k}\sim\mc{N}\left( \mbf{x}_{k}, \mbf{P}_{k} \right)$. Therefore, the parameters to be estimated should
    \begin{align}
        \tilde{\mbs{\theta}} &= \left\{ \mbf{x}_{0}, \ldots, \mbf{x}_{N}, \mbf{P}_{0}, \ldots, \mbf{P}_{N}\right\}.
    \end{align}
    However, to make things easier, the covariances $\mbf{P}_{0:N}$ will not be estimated (they will be computed after the MAP estimate is computed). Thus, the parameters that'll be estimated is given by
    \begin{align}
        \mbs{\theta} &= \left\{ \mbf{x}_{0}, \ldots, \mbf{x}_{N}\right\}.
    \end{align}
    The realizations are then
    \begin{align}
        \mbf{X} &= \left\{ \mbfcheck{x}_{0}, \mbf{u}_{0:N-1}, \mbf{y}_{1:N}\right\}.
    \end{align}
\end{myremark}
For the MAP estimator, the PDF of interest is
\begin{align}
    \pdf{\mbs{\theta}\middle|~ \mbf{X}} 
    \label{eq:MAP state estimation f(theta|X)}
    &= \pdf{\mbf{x}_{0}, \ldots, \mbf{x}_{N} \middle|~ \mbfcheck{x}_{0}, \mbf{u}_{0:N-1}, \mbf{y}_{1:N}} \\
    &= \f{\pdf{\mbf{y}_{1:N}\middle|~ \mbf{x}_{0:N}, \mbfcheck{x}_{0}, \mbf{u}_{0:N-1}}\pdf{\mbf{x}_{0:N}\middle|~\mbfcheck{x}_{0}, \mbf{u}_{0:N-1}}}{\pdf{\mbf{y}_{1:N}\middle|~ \mbfcheck{x}_{0}, \mbf{u}_{0:N-1}}}\\
    &= \eta \pdf{\mbf{y}_{1}\middle|~ \mbf{x}_{1}}\cdots\pdf{\mbf{y}_{N}\middle|~\mbf{x}_{N}}
        \pdf{\mbf{x}_{N}\middle|~\mbf{x}_{N-1}, \mbf{u}_{N-1}}\cdots\pdf{\mbf{x}_{1}\middle|~\mbf{x}_{0}, \mbf{u}_{0}}\pdf{\mbf{x}_{0} \middle|~ \mbfcheck{x}_{0}}\\
    \label{eq:MAP state estimation f(y|x)f(x|xm1)}
    &= \eta \prod_{k=1}^{N}\pdf{\mbf{y}_{k}\middle|~\mbf{x}_{k}}\prod_{k=1}^{N}\pdf{\mbf{x}_{k}\middle|~\mbf{x}_{k-1}, \mbf{u}_{k-1}} \pdf{\mbf{x}_{0}, \middle|~ \mbfcheck{x}_{0}},
\end{align}
where $\eta$ is a normalizing parameter that is independent of the design variables ($\mbf{x}_{0:N}$) and the following assumptions were used
\begin{enumerate}
    \item The measurements $\mbf{y}_{k}$ depends only on $\mbf{x}_{k}$ (and measurement noise $\mbfrv{n}_{k}$ )as shown in the measurement model \eqref{eq:MAP state estimation measurement model}. Thus, the random variable $\mbfrv{y}_{k}$ is independent of all other random variables. 
    
    \item The state $\mbfrv{x}_{k}$ depends only on $\mbfrv{x}_{k-1}$ and $\mbf{u}_{k-1}$ (and process noise $\mbfrv{w}_{k-1}$) as shown in \eqref{eq:MAP state estimation process model}. Therefore, the random variables $\mbf{x}_{0:N}$ follow the \textbf{Markov sequence}: the future states $\mbfrv{x}_{k+1:N}$ are independent of the past states $\mbfrv{x}_{0:k-1}$ given the present state $\mbfrv{x}_{k}$. The Markov sequence is dictated by the process model \eqref{eq:MAP state estimation process model}.
        
\end{enumerate}

If $\mbfrv{x}_{0:N}$ are assumed to be [marginally] Gaussian, then it is easier to minimize the negative-log of \eqref{eq:MAP state estimation f(theta|X)} than maximize \eqref{eq:MAP state estimation f(theta|X)} directly. 

Taking the negative-log of \eqref{eq:MAP state estimation f(y|x)f(x|xm1)} gives
\begin{align}    
    -\log \pdf{\mbf{x}_{0:N}\middle|~ \mbfcheck{x}_{0}, \mbf{u}_{0:N-1}, \mbf{y}_{1:N}}
    \label{eq:negative log likelihood}
    &=
    \f{1}{2}\norm{\mbf{x}_{0} - \mbfcheck{x}_{0}}^{2}_{\mbfcheck{P}_{0}} + 
    \f{1}{2}\sum_{k=1}^{N} \norm{\mbf{y}_{k} - \mbf{g}(\mbf{x}_{k}, \mbf{0})}^{2}_{\mbf{R}_{k}} +\\
    &\qquad  
    \f{1}{2}\sum_{k=1}^{N} \norm{\mbf{x}_{k} - \mbf{f}(\mbf{x}_{k-1}, \mbf{u}_{k-1}, \mbf{0})} + 
    \mbs{\gamma},
\end{align}
where 
\begin{align}
    \norm{\mbf{z}}^{2}_{\mbs{\Sigma}} &\triangleq \mbf{z}^{\trans}\mbs{\Sigma}\inv\mbf{z},
\end{align}
and
$\mbs{\gamma}$ are constant terms that are independent of the design variables thus they will not affect the optimization.

Therefore, the MAP estimate of $\mbfhat{x}_{0:N}^{\mathrm{MAP}}$ is the solution to 
\begin{align}
    \mbfhat{x}_{0:N}^{\mathrm{MAP}} &= \argmin_{\mbf{x}_{0:N}\in\rnums^{n}} \f{1}{2}\norm{\mbf{x}_{0} - \mbfcheck{x}_{0}}^{2}_{\mbfcheck{P}_{0}} + 
    \f{1}{2}\sum_{k=1}^{N} \norm{\mbf{y}_{k} - \mbf{g}(\mbf{x}_{k}, \mbf{0})}^{2}_{\mbf{R}_{k}} + 
    \f{1}{2}\sum_{k=1}^{N} \norm{\mbf{x}_{k} - \mbf{f}(\mbf{x}_{k-1}, \mbf{u}_{k-1}, \mbf{0})},
    &= \argmin_{\mbf{x}_{0:N}\in\rnums^{n}} J
\end{align}
which is a (nonlinear) weighted least squares problem. 

\begin{myremark}
    In \eqref{eq:negative log likelihood}, it was assumed that $\mbf{w}_{k} = \mbf{n}_{k} = \mbf{0}$. I'm not not sure why this assumption is done. The way I think of it is that the noise $\mbf{w}_{k}$ is embedded into the interoceptive measurement $\mbf{u}_{k}$. Similarly, the noise in $\mbf{n}_{k}$ is embedded in the exteroceptive measurement $\mbf{y}_{k}$.
\end{myremark}

\subsection{Expressing the nonlinear least squares problem in matrix form}
The objective function can be expressed as
\begin{align}
    J(\mbf{x}_{0:N}) 
    &= 
    \f{1}{2}\norm{\mbf{x}_{0} - \mbfcheck{x}_{0}}^{2}_{\mbfcheck{P}_{0}} + 
    \f{1}{2}\sum_{k=1}^{N} \norm{\mbf{x}_{k} - \mbf{f}(\mbf{x}_{k-1}, \mbf{u}_{k-1}, \mbf{0})} +
    \f{1}{2}\sum_{k=1}^{N} \norm{\mbf{y}_{k} - \mbf{g}(\mbf{x}_{k}, \mbf{0})}^{2}_{\mbf{R}_{k}} \\ 
    &=
    \f{1}{2}\mbf{e}(\mbf{x}_{0:N})^{\trans}\mbf{W}\mbf{e}(\mbf{x}_{0:N}),
\end{align}
where
\begin{align}
    \mbf{e}(\mbf{x}_{0:N}) &=
    \bbm
        \mbf{x}_{0} - \mbfcheck{x}_{0} \\ 
        \hline
        \mbf{x}_{1} - \mbf{f}\left( \mbf{x}_{0}, \mbf{u}_{0}, \mbf{0} \right) \\
        \vdots\\
        \mbf{x}_{N} - \mbf{f}\left( \mbf{x}_{N-1}, \mbf{u}_{N-1}, \mbf{0} \right) \\
        \hline
        \mbf{y}_{1} - \mbf{g}(\mbf{x}_{1}, \mbf{0})\\
        \vdots\\
        \mbf{y}_{N} - \mbf{g}(\mbf{x}_{N}, \mbf{0})\\
    \ebm
    \label{eq:MAP estimate Markov LS error function}
\end{align}
is the \emph{error function}, and
\begin{align}
    \mbf{W}\inv &= 
    \bbm
        \mbfcheck{P}_{0}    &   &   &   &   &   &   \\
            &  \mbf{Q}_{0} &   &   &   &   &   \\
            &   & \ddots  &   &   &   &   \\
            &   &   & \mbf{Q}_{N}  &   &   &   \\
            &   &   &   & \mbf{R}_{1}  &   &   \\
            &   &   &   &   & \ddots  &   \\
            &   &   &   &   &   & \mbf{R}_{N}  
    \ebm
    \label{eq:MAP estimate Markov LS weight matrix}
\end{align}
is the \emph{weight matrix}. 

\subsection{Covariance on MAP estimate}
In the optimization problem above, the covariance was not estimated. However, it can be estimated by using the MAP state estimates $\mbfhat{x}^{\mathrm{MAP}}_{0:N}$. The \emph{joint} covariance can be approximated by
\begin{align}
    \cov{\mbfhat{x}_{0:N}^{\mathrm{MAP}}} &=
    \left( \mbf{J}(\mbfhat{x}_{0:N}^{\mathrm{MAP}})^{\trans}\mbf{W}\mbf{J}\left( \mbfhat{x}_{0:N}^{\mathrm{MAP}} \right) \right)\inv,
\end{align}
where 
\begin{align}
    \mbfhat{x}_{0:N}^{\mathrm{MAP}}
\end{align}
is the Jacobian of the error function \eqref{eq:MAP estimate Markov LS error function} w.r.t. the design variables, evaluated at the MAP estimate $\mbfhat{x}_{0:N}^{\mathrm{MAP}}$. Specifically,
\begin{align}
    \mbf{J}(\mbfhat{x}_{0:N}^{\mathrm{MAP}}) &= 
    \left. \td{\mbf{e}\left( \mbf{x}_{0:N} \right)}{\mbf{x}_{0:N}}\right|_{\mbf{x}_{0:N}=\mbfhat{x}_{0:N}^{\mathrm{MAP}}}.
\end{align}

